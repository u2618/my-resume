\documentclass[singlesided,
               %doublesided,
               paper=a4,
               fontsize=10pt
              ]{my-resume}


%%%%%%%%%%%%%%%%%%%%%%%%%%%%%%%%%%%%%%%%%%%%%%%%%%%%%%%%%%%%%%%%%%%%%%%%%%%%%%%%
% set geometry
%%%%%%%%%%%%%%%%%%%%%%%%%%%%%%%%%%%%%%%%%%%%%%%%%%%%%%%%%%%%%%%%%%%%%%%%%%%%%%%%

\setlength\highlightwidth{8cm}
\setlength\headerheight{4cm}            % note that margintop gets added to this value, i.e. the header bar is 5cm
\setlength\marginleft{1cm}
\setlength\marginright{\marginleft}      % needs to be 1.5 times to be actually equal. why?
\setlength\margintop{1cm}
\setlength\marginbottom{1cm}


%%%%%%%%%%%%%%%%%%%%%%%%%%%%%%%%%%%%%%%%%%%%%%%%%%%%%%%%%%%%%%%%%%%%%%%%%%%%%%%%
% FONTS
%%%%%%%%%%%%%%%%%%%%%%%%%%%%%%%%%%%%%%%%%%%%%%%%%%%%%%%%%%%%%%%%%%%%%%%%%%%%%%%%

\RequirePackage{fontspec}
\setmainfont{Carlito}


%%%%%%%%%%%%%%%%%%%%%%%%%%%%%%%%%%%%%%%%%%%%%%%%%%%%%%%%%%%%%%%%%%%%%%%%%%%%%%%%
% COLORS
%%%%%%%%%%%%%%%%%%%%%%%%%%%%%%%%%%%%%%%%%%%%%%%%%%%%%%%%%%%%%%%%%%%%%%%%%%%%%%%%

\colorlet{highlightbarcolor}{lightgray}
\colorlet{headerbarcolor}{darkgray}

\colorlet{headerfontcolor}{white}
\colorlet{accent}{awesome-red}
\colorlet{heading}{black}
\colorlet{emphasis}{black}
\colorlet{body}{black}


%%%%%%%%%%%%%%%%%%%%%%%%%%%%%%%%%%%%%%%%%%%%%%%%%%%%%%%%%%%%%%%%%%%%%%%%%%%%%%%%
% set document
%%%%%%%%%%%%%%%%%%%%%%%%%%%%%%%%%%%%%%%%%%%%%%%%%%%%%%%%%%%%%%%%%%%%%%%%%%%%%%%%


\begin{document}

\name{Bruce Wayne}
\tagline{Amazing tag line that makes HR managers want to hire you immediately.\\The tag line is aligned with the bottom of the picture to the right, so if you have\\multiple lines it fills the space to your name from the bottom. You might want to\\introduce manual line breaks to make this look nicer then simply filling up the lines.}
\photo[round]{picture.jpg}{\dimexpr \headerheight-\marginbottom}   % make photo exactly match the header with margintop/marginright/marginbottom as margin

\makeheader

\highlightbar{

    \section{Contact}
    
    \email{mail@mailprovider.com}
    \phone{+49 123 456789}
    \location{Bat Cave 1, 12345 Gotham City}
    \homepage{Website}{https://www.batman.com}
    \github{@RealBatman}{https://github.com/RealBatman}
    \linkedin{Bruce Wayne}{https://www.linkedin.com/in/bruce-wayne-12345/}
    \orcid{0000-0000-0000-0000}{https://www.orcid.org/0000-0000-0000-0000}
    \ads{NASA/ADS publication list}{https://ui.adsabs.harvard.edu/search/fq=\%7B!type\%3Daqp\%20v\%3D\%24fq\_database\%7D&fq\_database=database\%3A\%20astronomy&p\_=0&q=pubdate\%3A\%5B2016-01\%20TO\%209999-12\%5D\%20author\%3A(\%22Wayne\%2C\%20Bruce\%22)&sort=date\%20desc\%2C\%20bibcode\%20desc}
    
    \section{Skills}
    
    \skillsection{Programming}
    \skill{Python}{5}
    \skill{Bash}{4}
    \skill{LaTeX}{4}
    
    \vspace{0.5em}
    \skillsection{Operating Systems}
    \skill{Linux}{4}
    \skill{MacOS}{5}
    \skill{Windows}{3}
    
    \vspace{0.5em}
    \skillsection{Software \& Tools}
    \skill{Visualisation}{5}
    \skill{Data handling/analysis}{5}
    \skill{Docker}{3}
    \skill{Office}{4}
    
    \vspace{0.5em}
    \skillsection{Another skill subsection header}
    You can also put simply text here without the dots.
    
    \vspace{0.5em}
    \skillsection{Languages}
    \skill{German}{5}
    \skill{English}{5}
    \skill{French}{2}
    \bigskip
    
    \section{Certificates}
    \simpleskill{AWS certified cloud practitioner}
    \bigskip
    
    \section{Awards}
    \cvachievement{\faTrophy}{Sexiest super hero alive}{Super hero academy, 2016}

}
\mainbar{
    \section{About this template}
    Section are set in bold face. An optional parameter of \texttt{\textbackslash section} takes a symbol to add in front of the text. This option is used in the jobs and education sections below.
    
    \section[\faCogs]{Work history}
    \cvevent{Job title}
            {Really long name of some employer I used to work for\\ Division within the really long employer name}
            {Date from -- Date to}
            {City}
    
    \begin{itemize}
    	\item Optional:
    	\item Details about your job
    \end{itemize}
    \divider 
    
    \cvevent{Job title}
    {Employer\\ Division }
    {Date} %you can skip either of the two fields
    {} %you can skip the city if it's in your company's name already
    \smallskip
    \tag{You}\tag{can}\tag{also}\tag{put}\tag{tags}\tag{here}\tag{to}\tag{link}\tag{to}\tag{your}\tag{skill section}
    
    \section[\faProjectDiagram]{Projects}
    \begin{cvtimeline}    
      \cvitemtime{Fighting the Joker}{Funded by Gotham City Super Hero Department}{02/2018}{}{} %by now, you get the idea behind cvevents
      
      \cvitemtime{Another Project}{Funded by Gotham City}{12/2012}{}{}
      
      \cvitemtime{And another one}{Funded by Gotham City}{07/1966}{Paramount}{Driving bad mobiles}

      \cvitemtime{}{}{03/1939}{}{}

    \end{cvtimeline}
    
    %%%% We stop here with filling up the page, because if we the content does not fit the page,
    %%%% the minipage will float onto the next page and essentially break the entire document.
 
}
\makebody
\clearpage


\pagestyle{highlightmain}
\highlightbar{ %you can leave this highlightbar empty if you use the doublesided option.
	           % If you are on single-sided, you need to put some content in here, otherwise
	           % the text of the mainbar will have a bad offset
	           % a simple \phantom{foo} will do the trick: It's some content, but it's invisible
	           
    \section{Some section}
    \cveventtight{Some event in the highlightbar}{Details}{Date}{Location} %other than \cvevent, \cveventtight will place date and location in two lines. There is not way to break lines, though. If you need that, edit the \cveventtight command in the class-file and look at how \cvref is implemented (with \begin{description}...\end{description}). These will break lines just fine but you need to fiddle with the spacing.
    
    \bigskip
    
    \section{Referenzen}
    
    \cvref{Prof. Dr. Important Guy}{Job title\\Your really long former employer whose name is too long for one line}{email@example.com}{Address Line 1\\Address line two}

} %leaving this completely empty will work as long as you have the doublesided option on. You do need to put some content in if you are on singlesided, though.
\mainbar{

    %%%% Here we continue with providing extra information if one side is not enough
    
    \section[\faGraduationCap]{Education}
    %education events work just like job events
    \cvevent{Master of Superheros}
    {University of Superheros}
    {Date}
    {Gotham City} 
      
    \section[\faBalanceScale]{Wheel Chart}
    % This is taken from AltaCV
    % see https://github.com/liantze/AltaCV for details
    \wheelchart{1.5cm}{0.5cm}{% outer and inner diameter
    	6/8em/accent!20/Sleep,          % comma-separated list of
    	8/8em/accent!40/Daytime job,    % fraction of 24 / line length / color / label
    	2/8em/accent!80/Training,          % here, the color is shades of the accent color
    	3/8em/accent!60/Recovering from fighting criminals,
    	5/8em/accent/Being Batman
    } 
    \section{Another section}
    
    This page uses the page style \texttt{highlightmain} which shows the highlight bar (gray) and the main part (white background) but omits the header. 
    The default page style is \texttt{headerhighlightmain} with all three elements.
    If you don't want header, nor highlight bar, use page style \texttt{\textbackslash pagestyle\{empty\}}.
    \medskip
    Neither main, nor highlight bar must be filled to make this template work.
    It is possible to use a page style with the highlight bar but leave it empty by setting an empty highlightbar \texttt{\textbackslash highlightbar\{\}}.

    \vspace{0.5em}
    \subsection{Subsection 1}
    Demonstrate subsections.
    
    \subsection{Subsection 2}
    Subsection are also bold face but a smaller font then section. They also omit the rule.
    

}
\makebody


\clearpage
\pagestyle{empty}

\section{Publications}
\pubforcefullwidth

Demonstrate what an \texttt{\textbackslash pagestyle\{empty\}} page looks like.
Also show off the macros for publications that uses small icons for authors, date, journal and links.

Achieving a good looking spacing can be tricky. For empty pagestyles where the full width is available use \texttt{\textbackslash pubforcefullwidth} to force the publoication list to take up all the available space.
The (relative) lengths reserved for date, journal and links can be set with the parameters \texttt{\textbackslash pubdatelength}, \texttt{\textbackslash pubjournallength} and \texttt{\textbackslash publinklength} as in \texttt{\textbackslash setlength\{\textbackslash pubdatelength\}\{0.15 \textbackslash linewidth\}}.
\bigskip

\publication
	{Fighting Criminals All Around the World} % Title
	{\textbf{B. Wayne}, S. Man, C. America, T. Thing} % Authors
	{2020} % Year
	{The Superhero Journal Vol. 1, Issue 1, id.1} % Journal
	{\ADS{https://ui.adsabs.harvard.edu/abs/...}, \arXiv{https://arxiv.org/abs/...}} % ADS & arxiv links


\end{document}
